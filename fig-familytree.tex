% !TEX root = 2p-kt-talk.tex
\begin{figure}
\tikzstyle{inactive}=[rectangle, draw=black, rounded corners, fill=gray!50, drop shadow,
        text centered, anchor=north, text=black]
\tikzstyle{active}=[rectangle, draw=black, rounded corners, fill=white, drop shadow,
        text centered, anchor=north, text=black]
\tikzstyle{myarrow}=[->, >=open triangle 90, thick]
\tikzstyle{line}=[-, thick]

\pgfdeclarelayer{background}
\pgfdeclarelayer{foreground}
\pgfsetlayers{background,main,foreground}
\usetikzlibrary{positioning}
\begin{adjustbox}{width=1\textwidth}    
    \begin{tikzpicture}[node distance=.6cm]
    % \node (Prolog0) [inactive, rectangle split, rectangle split parts=2]
    %     {
    %         \textbf{Prolog 0}
    %         \nodepart{second}{\scriptsize first Prolog System}
    %     };
    \node (Prolog1) [inactive, rectangle split, rectangle split parts=2]
        {
            \textbf{Prolog 0 \& I}
            \nodepart{second}{\scriptsize negation as failure}
        };
    % \draw[-Stealth] (Prolog0.south) -- (Prolog1.north) ;

    \node (Prolog2) [inactive, rectangle split, rectangle split parts=2, below=of Prolog1]
        {
            \textbf{Prolog II}
            \nodepart{second}{\scriptsize cyclic structures}
        };
    \draw[-Stealth] (Prolog1.south) -- (Prolog2.north) ;

    \node (Prolog3) [inactive, rectangle split, rectangle split parts=2, below=of Prolog2]
        {
            \textbf{Prolog III}
            \nodepart{second}{\scriptsize constraints}
        };
    \draw[-Stealth] (Prolog2.south) -- (Prolog3.north) ;

    \node (Prolog4) [inactive,
      % rectangle split, rectangle split  parts=2,
    below=of Prolog3]
        {
            \textbf{Prolog IV}
            % \nodepart{second}{\scriptsize constraints}
        };
    \draw[-Stealth] (Prolog3.south) -- (Prolog4.north) ;

    \node (DEC10) [inactive, rectangle split, rectangle split parts=2, right=of Prolog1, yshift=-3mm]
        {
            \textbf{DEC-10 Prolog}
            \nodepart{second}{\scriptsize compiled, de facto standard}
        };
    \draw[-Stealth] (Prolog1.east) -- (DEC10.west);

    \node (CProlog) [inactive, rectangle split, rectangle split parts=2, right=of DEC10, yshift=-2mm]
        {
            \textbf{C-Prolog}
            \nodepart{second}{\scriptsize interpreted, portable}
        };
    \draw[-Stealth] (DEC10.east) -- (CProlog.west);

    \node (WAM) [inactive, rectangle split, rectangle split parts=2, below=of DEC10, yshift=1mm]
        {
            \textbf{The WAM} % Warren Abstract Machine (WAM)
            \nodepart{second}{\scriptsize compiled, portable}
        };
    \draw[-Stealth] (DEC10.south) -- (WAM.north);
    \node (Quintus) [inactive, rectangle split, rectangle split parts=2, below=of WAM, yshift=2mm]
        {
            \textbf{Quintus}
            \nodepart{second}{\scriptsize commercial, de-facto standard}
        };
    \draw[-Stealth] (WAM.south) -- (Quintus.north);

    \node (SICStus) [active, rectangle split, rectangle split parts=2, below=of Quintus, yshift=3mm]
        {
            \textbf{SICStus}
            \nodepart{second}{\scriptsize commercial support, JIT}
        };
    \draw[-Stealth] (Quintus.south) -- (SICStus.north);

    \node (BIM) [inactive, rectangle split, rectangle split parts=2, right=of Quintus, xshift=6mm, yshift=-4mm]
        {
            \textbf{BIM} % -Prolog
            \nodepart{second}{\scriptsize commercial, native}
        };
    \draw[-Stealth] (WAM.east) -| (BIM.north);

    \node (Ciao) [active, rectangle split, rectangle split parts=2, below=of SICStus, yshift=3mm]
        {
            \textbf{\&-Prolog/Ciao}
            \nodepart{second}{\scriptsize parallel, assertions}
        };
    \draw[-Stealth] (SICStus.south) -- (Ciao.north);

    \node (SWI) [active, rectangle split, rectangle split parts=2, right=of Ciao]
        {
            \textbf{\ SWI\ } % SWI-Prolog
            \nodepart{second}{\scriptsize libraries}
        };
    \draw[-Stealth] (Quintus.south east) -- (SWI.north);

    \node (YAP) [active, rectangle split, rectangle split parts=2, right=of SWI]
        {
            \textbf{YAP} % -Prolog
            \nodepart{second}{\scriptsize indexing}
        };
    \draw[-Stealth] (Quintus.east) -- (YAP.north west);

    \node (SB) [inactive,
    % rectangle split, rectangle split parts=2,
    right=of YAP]
        {
            \textbf{SB-Prolog}
            \nodepart{second}
        };
    \draw[-Stealth] (WAM.east) -| (SB.north);

    \node (XSB) [active, rectangle split, rectangle split parts=2, below=of SB]
        {
            \textbf{XSB}
            \nodepart{second}{\scriptsize tabling}
        };
    \draw[-Stealth] (SB.south) -- (XSB.north);

    \node (GNU) [active, rectangle split, rectangle split parts=2, right=of XSB]
        {
            \textbf{GNU} % -Prolog
            \nodepart{second}{\scriptsize fd/indexicals}
        };
    \draw[-Stealth] (WAM.east) -| (GNU.north);

    \node (OtherW) [active, right=of GNU, xshift=-2mm]
        {
            \textbf{\ \ \ldots\ \ }
        };

    \node (BProlog) [active, rectangle split, rectangle split parts=2, left=of XSB]
        {
            \textbf{B-Prolog}
            \nodepart{second}{\tiny TOAM}
        };
    \draw[-Stealth] (SB.south west) -- (BProlog.north east);

    \node (Bin) [active, rectangle split, rectangle split parts=2, left=of BProlog, xshift=3mm]
        {
            \textbf{BinProlog}
            \nodepart{second}{\scriptsize binarization}
        };

    \node (tuProlog) [active, rectangle split, rectangle split parts=2, left=of Bin, xshift=3mm]
        {
            \textbf{tuProlog}
            \nodepart{second}{\scriptsize jvm, interop.} % interoperability
        };

    \node (OtherNW) [active, below=of BProlog, xshift=3mm, yshift=4mm]
        {
            \textbf{\ \ \ldots\ \ } 
        };

    \node (Marseille1) [below= 2mm of Prolog4]
        {
            \textbf{Marseille}
        };
    \node (Marseille2) [below= -1mm of Marseille1]
        {
            \textbf{Prolog-line}
        };

    \node (WAM-comment2) [right=of Quintus, xshift=-2mm, yshift=4mm]
        {
            \textbf{Prologs}
        };
    \node (WAM-comment1) [above=0mm of WAM-comment2]
        {
            \textbf{WAM-based}
        };
    % \node (WAM-comment) [below= 1mm of GNU, align=right,xshift=-6mm]
    %     {
    %         \textbf{WAM-based Prologs}
    %     };

    \node (NWAM-comment) [below=1mm of tuProlog,align=right,xshift=5mm]
        {
            \textbf{WAM alternatives}
        };

        \begin{pgfonlayer}{background}
        \path (Prolog1.west |- Prolog1.north)+(-0.2,0.2) node (a) {};
        \path (Marseille2.east |- Marseille2.south)+(+0.3,0) node (c) {};
        \path[fill=gray!5,rounded corners, draw=black!50, dashed]
              (a) rectangle (c);

        \path (Quintus.west |- WAM.north)+(-0.2,0.2) node (wam-a) {};
        \path (OtherW.east |- GNU.south)+(+0.2,-0.2) node (wam-c) {};
        \path[fill=gray!5,rounded corners, draw=black!50, dashed]
              (wam-a) rectangle (wam-c);

        \path (tuProlog.west |- tuProlog.north)+(-0.2,0.2) node (nwam-a) {};
        \path (BProlog.east |- NWAM-comment.south)+(+0.3,-0.1) node (nwam-c) {};
        \path[fill=gray!5,rounded corners, draw=black!50, dashed]
              (nwam-a) rectangle (nwam-c);
        \end{pgfonlayer}
    \end{tikzpicture}
\end{adjustbox}
\end{figure}
